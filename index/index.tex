%INPUT
\documentclass[10pt, a4paper,ngerman]{article}
\usepackage{babel}
\usepackage{titlesec}
\usepackage[T1]{fontenc}
\usepackage[utf8]{inputenc}
\usepackage[a4paper, left=6mm,right=6mm, top=6mm, bottom=6mm]{geometry}
\special{papersize=210mm,297mm}

%MATHEMATIK
\usepackage{amssymb}
\usepackage{lmodern}
\usepackage{amsmath}
\usepackage{cancel}
\usepackage{enumerate}
\usepackage{esint}
\usepackage{empheq}
\usepackage{lmodern}
\usepackage[thickspace,squaren,textstyle]{SIunits} 
\DeclareMathOperator{\arsinh}{arsinh}
\DeclareMathOperator{\arccot}{arccot}

%LAYOUT
\usepackage[T1]{fontenc}
\usepackage{charter}
\usepackage{float}
\usepackage[margin=1cm, justification=centering, singlelinecheck=no,tablename=Tab., figurename=Abb.]{caption}
\usepackage{comment}
\usepackage{graphicx}
%\usepackage{hyperref}
\usepackage{xcolor}
\usepackage{wrapfig}
\usepackage{multicol}

% BOLD
\let\oldtextbf\textbf
\renewcommand{\textbf}[1]{\textcolor{violet}{\oldtextbf{#1}}}

% ITALIC
%\let\oldtextit\textit
%\renewcommand{\textit}[1]{\textcolor{orange}{\oldtextit{#1}}}

% COLORS
\usepackage{sectsty}
\chapterfont{\color{violet}}
\sectionfont{\color{purple}}
\subsectionfont{\color{blue}}
\subsubsectionfont{\color{violet}}

%BOXED
\newcommand*{\boxedcolor}{red}
\makeatletter
\renewcommand{\boxed}[1]{
  \textcolor{\boxedcolor}{%
  \fbox{\normalcolor\m@th$\displaystyle#1$}}
}

%BOXED_COLORED
\usepackage{tcolorbox}
\definecolor{mycolor}{rgb}{0.122, 0.435, 0.698}
\newtcbox{\boxi}{on line,
  colframe=mycolor,colback=cyan!10!white,
  boxrule=0.5pt,arc=1pt,boxsep=0pt,left=6pt,right=6pt,top=3pt,bottom=3pt}

\usepackage{fancyhdr} 
\fancypagestyle{style1}{
  \fancyhf{}
  \fancyhead[C]{style 1 with thin line}
  \fancyfoot[C]{\thepage}
  \renewcommand{\headrulewidth}{0.4pt}
}

% \pagestyle{style2}
\usepackage{lipsum}

% LSTLISTINGS
\usepackage{listings}

% Code einbinden
\lstset{numbers=left, numberstyle=\tiny, numbersep=10pt} \lstset{language=Java} 
\lstdefinestyle{customc}{
  belowcaptionskip=1\baselineskip,
  breaklines=true,
  frame=L,
  xleftmargin=\parindent,
  language=C,
  showstringspaces=false,
  basicstyle=\footnotesize\ttfamily,
  keywordstyle=\bfseries\color{green!40!black},
  commentstyle=\itshape\color{purple!40!black},
  identifierstyle=\color{blue},
  stringstyle=\color{orange},
}
\lstdefinestyle{customasm}{§
  belowcaptionskip=1\baselineskip,
  frame=L,
  xleftmargin=\ parindent,
  language=[x86masm]Assembler,
  basicstyle=\footnotesize\ttfamily,
  commentstyle=\itshape\color{purple!40!black},
}
\lstset{escapechar=@,style=customc}

% DOCUMENT
\begin{document}
\thispagestyle{empty}

\begin{flushleft}
  \textit{RAMÍREZ ROBAYO Oscar Mauricio}
  \\
  %\textcolor{blue}{|}
  \textit{Digital Engineering Robotik \& Big Data}
  \\
  \textcolor{blue}{\textbf{\textit{Git und GitHub}}}
  \\
  \textcolor{violet}{{\rule{\paperwidth/2}{1pt}}}
\end{flushleft}
\section{Git}
\%Username\\
\texttt{\$ git config --global user.name ``Oscar Ramirez''}\\\\
\%Mail\\
\texttt{\$ git config --global user.email ``or473338@gmail.com''}\\\\
\%Branches\\
\texttt{\$ git config --gloal init.default branch main}\\\\
\%Help \\
\texttt{\$ git config -h}\\\\
\%Command help\\
\texttt{\$ git help config}\\\\
\%Change Folder for repository\\
\texttt{\$ cd c:/users/or473/OneDrive/Git-GitHub}\\\\
\%Initialize empty git repository\\
\texttt{\$ git init}\\\\
\%Status of repository with untracked files in what will be committed\\
\texttt{\$ git status}\\\\
\%Add index.html for being committed\\
\texttt{\$git add index.html}\\\\
\%Remove index.html for being removed\\
\texttt{\$ git rm --cached index.html}\\\\
\%All files will be tracked\\
\texttt{\$ git add .}\\\\
\%First commit\\
\texttt{\$ git commit -m ``First commit - committing all files to the repository''}\\\\
\%Add and commit all files\\
\texttt{\$ git commit -a -m ``updated the home.html site''}\\\\
\%Show the difference between the old and the new file\\
\texttt{\$ git diff}\\\\
\%Deleting a file\\
\texttt{\$ git rm ``home.html''}\\\\
\%Restore a file\\
\texttt{\$ git restore --staged ``home.html''}\\\\
\%Recovering a file\\
\texttt{\$ git restore ``home.html''}\\\\
\%Rename a file\\
\texttt{\$ git mv ``ReadMe.txt'' ``Leeme.txt''}\\\\
\%Show log of all commits\\
\texttt{\$ git log}\\\\
\%Show log of all commits in one line\\
\texttt{\$ git log --oneline}\\\\
\%Correct the commit message\\
\texttt{\$ git commit -m ``Fourth commit - renaming the file ReadMe.txt'' --amend}\\\\
\%Jump back to a commit\\
\texttt{\$ git reset 3a9d53a}\\\\
\%Rebase all order of the commits
\texttt{\$git rebase -i --root}\\\\
\%Create a new branch\\
\texttt{\$ git branch FixHome}\\\\
\%See a list of all branches\\
\texttt{\$ git branch}\\\\
\%Change branch\\
\texttt{\$ git switch FixHome}\\\\
\%Merge a file back to a certain branch
\texttt{\$ git merge -m ``Merge fixHome back to master'' FixHome}\\\\
\%Delete a branch\\
\texttt{\$ git branch -d FixHome}\\\\
\%Create and switch to a branch\\
\texttt{\$ git switch -c UpdateText}\\\\
\%Merge updated text between branches, MERGING, deleting a branch in between\\
\texttt{\$ git commit -a -m ``update text on index''}\\\\
\section{GitHub}
\texttt{git init}\\
\texttt{git add README.md}\\
\texttt{git commit -m "first commit"}\\
\texttt{git branch -M main}\\
\texttt{git remote add origin https://github.com/OscarRamirez473338/Git-GitHub.git}\\
\texttt{git push -u origin main}\\\\
Or push an existing repository from the command line\\
\texttt{git remote add origin https://github.com/OscarRamirez473338/Git-GitHub.git}\\
\texttt{git branch -M main}\\
\texttt{git push -u origin main}\\
\newpage
\begin{center}
  \includegraphics[scale=0.62]{../pngeggmejorada.png}
\end{center}

\end{document}